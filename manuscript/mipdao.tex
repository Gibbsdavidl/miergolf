%%%%%%%%%%%%%%%%%%%%%%%%%%%%%%%%%%%%%%%%%
% Journal Article
% LaTeX Template
% Version 1.3 (9/9/13)
%
% This template has been downloaded from:
% http://www.LaTeXTemplates.com
%
% Original author:
% Frits Wenneker (http://www.howtotex.com)
%
% License:
% CC BY-NC-SA 3.0 (http://creativecommons.org/licenses/by-nc-sa/3.0/)
%
%%%%%%%%%%%%%%%%%%%%%%%%%%%%%%%%%%%%%%%%%

%----------------------------------------------------------------------------------------
%	PACKAGES AND OTHER DOCUMENT CONFIGURATIONS
%----------------------------------------------------------------------------------------

\documentclass[twoside]{article}

\usepackage{natbib}
\usepackage{lipsum} % Package to generate dummy text throughout this template

\usepackage[sc]{mathpazo} % Use the Palatino font
\usepackage[T1]{fontenc} % Use 8-bit encoding that has 256 glyphs
\linespread{1.05} % Line spacing - Palatino needs more space between lines
\usepackage{microtype} % Slightly tweak font spacing for aesthetics

\usepackage[hmarginratio=1:1,top=32mm,columnsep=20pt]{geometry} % Document margins
\usepackage{multicol} % Used for the two-column layout of the document
\usepackage[hang, small,labelfont=bf,up,textfont=it,up]{caption} % Custom captions under/above floats in tables or figures
\usepackage{booktabs} % Horizontal rules in tables
\usepackage{float} % Required for tables and figures in the multi-column environment - they need to be placed in specific locations with the [H] (e.g. \begin{table}[H])
\usepackage{hyperref} % For hyperlinks in the PDF

\usepackage{lettrine} % The lettrine is the first enlarged letter at the beginning of the text
\usepackage{paralist} % Used for the compactitem environment which makes bullet points with less space between them

\usepackage{abstract} % Allows abstract customization
\renewcommand{\abstractnamefont}{\normalfont\bfseries} % Set the "Abstract" text to bold
\renewcommand{\abstracttextfont}{\normalfont\small\itshape} % Set the abstract itself to small italic text

\usepackage{titlesec} % Allows customization of titles
\renewcommand\thesection{\Roman{section}} % Roman numerals for the sections
\renewcommand\thesubsection{\Roman{subsection}} % Roman numerals for subsections
\titleformat{\section}[block]{\large\scshape\centering}{\thesection.}{1em}{} % Change the look of the section titles
\titleformat{\subsection}[block]{\large}{\thesubsection.}{1em}{} % Change the look of the section titles

\usepackage{fancyhdr} % Headers and footers
\pagestyle{fancy} % All pages have headers and footers
\fancyhead{} % Blank out the default header
\fancyfoot{} % Blank out the default footer
\fancyhead[C]{Running title $\bullet$ November 2012 $\bullet$ Vol. XXI, No. 1} % Custom header text
\fancyfoot[RO,LE]{\thepage} % Custom footer text

%----------------------------------------------------------------------------------------
%	TITLE SECTION
%----------------------------------------------------------------------------------------

\title{\vspace{-15mm}\fontsize{24pt}{10pt}\selectfont\textbf{Pathway
    analysis using the maximal influence problem solved by ant optimization}} % Article title

\author{
\large
\textsc{David L Gibbs}\thanks{Correspondence can be addressed to gibbsd@ohsu.edu}\\[2mm] % Your name
\normalsize OHSU \\ % Your institution
\normalsize \href{mailto:gibbsd@ohsu.edu}{gibbsd@ohsu.edu} % Your email address
\vspace{-5mm}
}
\date{}

%----------------------------------------------------------------------------------------

\begin{document}
\maketitle 
\thispagestyle{fancy} % All pages have headers and footers

%----------------------------------------------------------------------------------------
%	ABSTRACT
%----------------------------------------------------------------------------------------

\begin{abstract}

\noindent Here is my abstract. It's about a method for analyzing the
relationships of reactions in a pathway. In particular the influence
one reaction has over connected reactions in a pathway and how this
might relate to disease.

\end{abstract}

%----------------------------------------------------------------------------------------
%	ARTICLE CONTENTS
%----------------------------------------------------------------------------------------

\begin{multicols}{2} % Two-column layout throughout the main article text

\section{Introduction}

\lettrine[nindent=0em,lines=2]{C}ells recieve cues from the external environment, transmit the
information to the cell interior, and process the information, and
respond to it. The information is carried molecularly along paths.

PATHWAYS. Biological pathways are directed graphs describing branching sequences of molecular
interactions. Pathways can describe different types of biological
processes as well including signalling (phosphorylation) pathways,
gene regulation pathways, metabolic and chemical synthesis pathways,
and mixtures. Within pathways, there are a number of different types
of interactions possible including activations, repressions, carried out by means of
phosphorilations, sumoylations. But, essentially, in a large number of
cases, it is actually informaiton that is being transmitted.

FLOW NETWORKS. Flow methodologies in networks have found good sucess with methods
such as Hotnet, responsenet, resistor networks, and others. 
In the previous work on information flow in interaction networks, two
different models were considered, one of absorbtion, and one of
emitting. However, to get a full picture of how a network component
operates in information flow, it seems necessary to take both
views. The component both absorbs incoming infomration from the
upstream components, and in turn transmits this infomration to the
ddown stream components. 

LINE GRAPHS. However, if we think about what pathway component to focus on, clearly
in past work, the focus has been on vertices. Largely, PPI networks
have been considered (which are non-directional) and genes/proteins
are the focus. With the exception of hotnet that attempts to discover
connected subgraphs. But in the pathway, It is not a single entitiy
that might be considered functionally important -- but rather an
iteraction between a pair of entities. The interaction is an event
that needs to happen before downstream events take place, and is
caused by upstream events. In this work then I invert the normal
focus, and intead look at the reletive importance of events in the
pathway that have influence over other events.

DYNAMICS and CONTEXT. As the topology of biological networks is not regular, and much of
dynamic biology is largely contextual, the components of the pathway have
different levels of importance given any particular time and
state. For example, a given gene may be active (dynamic) in some
contexts, such as environmental change, but not others.
So, depending on the context, injury to some pathway components would
be relatively harmless, other pathways could make up for the difference. But in some circumstances,
injury to the component would possible have larger effects in the
network, possibly leading to disease.
Injury to a component might happen in a number of different ways
including mutations to the DNA, miRNA interference, infection
interference, a protein misfolding.

CONTEXT from DATA. To make the analysis contexutally sensitive, we turn to the use of
data. By applying data taken representing biological contrasts, such
as case and control studies, we can make the analysis reflective of
the active biological processes at work. Certainly, in reponse to
different biological states, certain pathways will be activated, or
disregulated. To do this we map the data to nodes on the pathway, and
apply a defined ``integration function''. The integration function
tells us how to approach using multiple data types, and puts a weight
on the edge connecting entities in the pathway. This might represent
how likely it is that the event was in fact activated.

INFLUENCE. Another way to explore the idea of importance in pathway components is by
considering influence in a directed graph. Given a particular context,
and supported by data, a component
might be considered important if it exercises influence over a large neighborhood
on the graph. 

QUANTIFY INFLUENCE. To measure influence is by using information diffusion
models. Information diffusion on graphs can be modeled as a random
walks on directed graphs. Certain, defined nodes in the graph are
considered sources of information. Information diffuses (walks) from
node to node probabilistically. Levels of dampening on information transmitting between nodes
can be tuned to reveal a localized view of influence. The dampening
can represent a model of unknown interferences. 

SUPERPATHS. This method should be applicable to very large super-pathways which
are constructed by joining smaller described pathways. This actually
brings a very new way to determine the influence of one pathway on
neighboring pathways, and how injury to influential pathway events
transmit to adjacent pathways, affecting the network at large.

MAXIMUM INFLUENCE PROBLEM. In such large pathways, it's expected that there would be many or
multple influential edges. Even with a metric for measuring the
influcene of a particular event in the graph, using data, it's a hard
combinatorial problem to choose the best set of pathway events that
have the greatest cumulative influence in the graph. This problem is
congruent with choosing K number of nodes to maximize influence. This
problem has been discussed in other domains such as social
networking. 

ANT OPTIMIZATION. A way to develop good solutions is by using the meta-heuristic of ant
colony optimization which has been shown to be very strong at solving
combinatorial problems such as the travelling salesmen problem and
subset selection problems like the knapsack problem. In this work, we
are solving a knapsack type problem, where K number of pathway edges
are selected to maximize the area of influence.


%------------------------------------------------

\section{Methods}

The mathematical description of the anlaysis framework follows.

\subsection{Data Input and Edge weight computation}

Pathways are encoded as a graph, $G = \langle V, E \rangle$, a set of vertices $V$,
connected by edges, $E$. Each edge can be described as $E_{ij} = (V_i ,
V_j, w_{ij}, t_{ij})$ where the directed edge connects from $V_i$ to $V_j$,
and carries weight $w_{ij} \in \mathbb{R}$ with type $t_{ij} \in \mathbb{N}$ that describes the
type of pathway edge.

Vertices represent biological entities such as genes, proteins,
functional miRNAs, and so on. Edge weights are computed using an
predefined integration function, $\mathcal{F}$, and data
mapped to vertices, notated as $D(V_i)$ for data mapped to vertice
$V_i$. The integration function will largely depend on the type of
data at hand, and the type of pathway edges considered,
but one example (that disregards the edge type) 
would be the mutual information function.

\begin{equation}
\label{integration-function}
w_{ij} = F(D(V_{i}) , D(V_{j})) = \mathcal{I} (D(V_{i}) ; D(V_{j}))
\end{equation}

Integration functions give confidence to the notion that
under the biological context, the pathway edge, or event, in question is active. 

\subsection{Directed line graph transformation}

Line graphs, also known as edge graphs, show the adjacency of edges to 
one another. The line graph, $G'$ is derived using $G$, as described
above. The new, directed, graph is notated as $G' = \langle V', E' \rangle$ where
the vertices now represent edges in $G$, $V' \in E$ and the edges of the line
graph, $E' \in E \times E$ are written as 

\begin{equation}
\label{line-graph-edges}
E'_{ij} = (E_i, E_j, \mathit{f} (W(E_i), W(E_j)))
\end{equation}

Where edge, $E_i \in G$ leads into $E_j \in G$, $W(E_i)$ is the edge
on the edge in $G$, and $\mathit{f}$ is some function on the weights,
a product is used in this work.

For two verices in the line graph to be connected, the edge indices
in $G$ must be a sequence. For example, from $E(V_1, V_2)$ to $E(V_2,
V_3)$ forms a sequence and would be represented in the line graph as
being connected. But, two edges (vertices in the line graph) from $E(V_1, V_2)$ to $E(V_3,
V_2)$ would not be connected.
 
\subsection{The Radio Model for information broadcast}

The radio model describes a broadcast area, or area of influence,
given a vertice on the directed graph. The vertice both recieves
infomration and transmits information. Information can only flow
in the direction of the given directed edges.

The radio model is built from two parts, the absorbing and emitting
models first described in work from Yu and Stojmirovi\'{c} \citep{Stojmirovic:2007eg}.
 
The markov transition matrix $P_{ij}$ defines the probability of
moving from one node, $V' \in G'$, to another. $P$ is computed by

\begin{equation}
\label{transition-matrix}
P_{ij} = \frac{W(E'_i)} {\sum W(E')}
\end{equation}


We choose a set $S$

The absorbtion model $G = (I - P_{tt})^{-1} P_{ts}$

The emitting model $H = P_{st} (I - P_{tt})^{-1}$

\subsection{Maximum Influence Problem}

The maximum influence problem seeks to find K nodes which maximize
some score using diffusion models.

\subsection{Hypercube Min-Max Ant Optimization for subset problems}

For ant colony optimization, transform the line graph to an 
all-to-all graph with tuple
weights (x,y) where x is the edge weight from the integration
function, and y is the ant pheromone.

Hypercube Min-max ant colony optimization is used to solve the
Maximal Influence Problem on $k$ vertices (reactions in the pathway).

The set of influential edges in the pathway are ranked by the
diffusion model. Edges with greater diffusion are ranked first.
The information diffusion model is based upon a random walk on the
graph, giving long term estimates of information absorbtion on a node,
and the amounts of infomration emitted.

The result of the method is a ranked list of pathway edges,
representing the most important and influential reactions in the system.


%------------------------------------------------

\section{Results}

Some results here.


%------------------------------------------------

\section{Discussion}

\subsection{Influential edges}

In this portion we discuss the most influential reactions, and why
they're so influential.

\begin{table}[H]
\caption{Diffusion Models}
\centering
\begin{tabular}{llr}
\toprule
\multicolumn{2}{c}{Name} \\
\cmidrule(r){1-2}
Model & Score & Top Hit \\
\midrule
Linear & 1 & etcm \\
Probabilistic & 3 & ggtt \\
\bottomrule
\end{tabular}
\end{table}


\subsection{Cutting edges to produce sub-graphs}

It may be possible to take the ranked edges, and cut them, producing a
set of sub-graphs containing nodes that are functionally related and
show the redundancy in the system.

\subsection{Relationships to information flow}

I am very curious as to whether information flow relates to the
influence score in graphs.  This might change with different models of 

%----------------------------------------------------------------------------------------
%	REFERENCE LIST
%----------------------------------------------------------------------------------------

\bibliographystyle{unsrt}
\bibliography{mipdao}

%----------------------------------------------------------------------------------------

\end{multicols}

\end{document}
